\section*{Abstract}

Gene duplication events are widely recognized as a key factor in the evolution of organismal complexity.
Mirroring broader contrasts between adaptation- versus contingency-driven hypotheses for the evolutionary origins of biological complexity, gene duplication outcomes are typically framed in terms of neo- and sub-functionalization scenarios.
In the former, duplicated genetic material catalyzes novel functionality; in the latter, it is co-opted to elaborate existing functionality.
Examples of both scenarios are widespread in natural history, but practical constraints have limited direct experimental investigation of the relationship between gene duplication and organismal complexity.
Using the Avida platform for digital evolution, we show that while increased genome size can promote the emergence of simple adaptive traits, gene duplication uniquely facilitates the \textit{de novo} evolution of complex adaptive phenotypes.
Tracing the ancestry of individual genetic sites, we find that slip duplication of a site increases its subsequent likelihood to code for novel phenotypic traits.
We then harness the unique \textit{in silico} capabilities of our model system to compare evolutionary outcomes across degraded variants of full-fledged gene duplication.
This ablative analysis confirms that the observed adaptive potentiation indeed arises from the duplication of existing genetic information.
In contrast to purely neutral framings of biological complexity, our results support gene duplication events as a contributing factor in adaptive origins of complex traits.
