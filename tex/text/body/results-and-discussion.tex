\section{Results and Discussion} \label{sec:results}

\subsection{Gene duplication facilitates adaptive evolution of complex traits}

In a first set of experiments, we investigated the impact of slip duplications on \textit{de novo} evolution of adaptive phenotypic traits.
Comparing the acquisition of Logic-9 traits between slip-duplicate-enabled and baseline treatments, we found slip duplications to yield significantly higher evolved phenotypic adaptation scores (two-tailed Mann-Whitney U test, $W = 562.5$, Bonferroni-adjusted $p << 0.0001$; Figure \ref{fig:results_panels}).
Investigating further, we noticed that most replicates saw substantial growth in genome length --- in some cases, a ten-fold increase up to sizes slightly more than 1,000 sites.
To test whether adaptive benefit was attributable to genome length, we performed additional experiments incorporating a ``long-genome'' control treatment initialized using genomes extended to 1,000 sites with neutral inserts.

\begin{figure*}
    \centering
    \includegraphics[width=\linewidth]{binder/binder/teeplots/adaptive-evolution-rate.ipynb/col=task-complexity+errorbar=se+hue=treatment+kind=line+mutation=per site+post=plt-xlim-0-600+style=treatment+viz=relplot+x=generation+y=has-task+ext=.pdf}
    \caption{
        \textbf{Gene duplication boosts adaptive evolution of complex phenotypic traits.}
        \footnotesize
        Plots show fraction of replicates exhibiting available phenotypic traits, by generation from founding ancestor.
        Panels facet by trait complexity, measured by the minimum number of NAND operations required to complete the task.
        Simple tasks (top left) require only one NAND operation.
        More complex tasks require up to five NAND operations (bottom right).
        Slip-duplication treatment facilitates significantly faster adaptive evolution than long-genome treatment for more complex tasks, requiring 4 or 5 components.
        Error bands give 95\% CI, bootstrapped over 30 replicates per treatment.
    }
    \label{fig:adaptive-evolution-rate}
\end{figure*}


Comparing the long-genome and baseline treatments, we observed increased genome length alone to also significantly boost task acquisition.
Disaggregating by task complexity, though, reveals impact of genome length as most prominent in the acquisition of simple tasks.
Figure \ref{fig:adaptive-evolution-rate} compares acquisition rates for tasks across task complexity classes --- with and without slip duplication, including the long-genome control.
The long-genome control matched or exceeded the performance of slip-duplication in evolving simple traits with 3 or fewer components.
% https://github.com/chaynes2019/AvidaGeneDupe/blob/538ede79c7301f10718ca96c8dd38782b6882632/binder/adaptive-evolution-rate.ipynb
However, slip duplication evolved more complex 4- and 5-component traits within a significantly higher fraction of replicates compared to the long-genome control (Fisher's exact tests; 36/60 vs. 24/60, $p<0.05$ [4 components]; 10/30 vs. 2/30, $p<0.03$ [5 components]; Figure \ref{fig:adaptive-evolution-rate}).

These results align with existing findings across a wide variety of biological taxa and digital models that slip-duplication of genetic material can facilitate evolution of adaptive traits \citep{Koza:1995fr,Zhang:2003fw,Teichmann:2004cz}.
Indeed, a striking example of gene duplication leading to new adaptations comes from the Long-Term Evolution Experiment in \textit{Escherichia coli}, where a duplication in one population broadened expression of key metabolic machinery to previously inhibitive conditions --- resulting in a 7-fold increase in population size \citep{blount_genomic_2012}.
Although a longstanding hypothesis dating as far back as Ohno \citep{ohno1970evolution}, however, results connecting experimental evidence connecting gene duplication, in particular, to complex traits is less well-established.

% are specific to adaptive evolution
\subsection{Information content of duplications provides adaptive benefit}

\begin{figure}[!h]
  \centering
  \begin{adjustbox}{scale=0.8}
    \includegraphics[
        height=4in,
        trim={0.2cm -1.5cm 0.2cm 0},
        clip
      ]{binder/binder/teeplots/col=split+env=static+hue=treatment+inner=box+kind=violin+palette=muted+viz=catplot+x=treatment+y=tasks-present+ext=.pdf}%
    \hspace*{-2.0cm}%
    \raisebox{0.125in}{%
      \includegraphics[
          height=2.7in,
          trim={1.36cm -0.64cm 0 0},
          clip
        ]{binder/binder/teeplots/env=static+errorbar=ci+hue=treatment+kind=line+palette=muted+viz=relplot+x=time-100k+y=tasks-present+ext=.pdf}}%
  \end{adjustbox}

  \vspace{-7ex}

  \begin{subfigure}{0.3\textwidth}
    \caption{\small slip-duplication}
    \label{fig:results_panels:slip_duplication}
  \end{subfigure}%
  \begin{subfigure}{0.35\textwidth}
    \caption{\small ablation treatments}
    \label{fig:results_panels:ablation}
  \end{subfigure}%
  \begin{subfigure}{0.22\textwidth}
    \caption{\small lineage history}
    \label{fig:results_panels:time_series}
  \end{subfigure}

  \vspace{1ex}

  \caption{\textbf{Treatments preserving slip-duplicated content facilitate adaptive evolution.}
    \small Violin plots show number of adaptive traits evolved in final dominant genotypes.
    Time series (\ref{fig:results_panels:time_series} right) shows progression of adaptive phenotypic trait counts along lineages of final dominant genotypes; color-coding corresponds to violin plots.
    Asterisk (*) markers indicate treatments with significantly more adaptive phenotypic traits compared to baseline, comparison across both \ref{fig:results_panels:slip_duplication} and \ref{fig:results_panels:ablation} panels.
    Simulation time unit is “updates,” corresponding to evaluation of ~30 genome sites per organism.}
  \label{fig:results_panels}
\end{figure}


Having observed that slip-duplicate mutations accelerate evolution of Logic-9 phenotypes, we next sought to isolate the aspects of slip duplication contributing to adaptation.
For this purpose, we tested four variants of the slip-duplication operator, disabling or replacing a particular aspect of slip duplication (overviewed in Figure \ref{fig:slip_mut_variants}).

As shown in Figure \ref{fig:results_panels}, we detected benefits to adaptive evolution only for the follow-up slip-scramble treatment --- which randomized sequence order within duplicated regions (two-tailed Mann-Whitney U test; Figure \ref{fig:results_panels}).
Phenotypic adaptation scores under all other slip-duplicate variants were indistinguishable or slower-adapting compared with the baseline treatment.

Given the efficacy of the slip-scramble treatment in facilitating adaptation, we additionally tested whether phenotypic adaptation differed between the slip-scramble and full-fledged slip-duplicate operators.
To prevent issues with multiple comparisons, we dispatched 100 new trials under both treatments for this test.
% p = 0.011
We found that the slip-duplicate treatment did, in fact, yield higher task counts compared to the slip-scramble treatment (two-tailed Mann-Whitney U test, $W = 4305$ respectively, Bonferroni-adjusted $p < 0.02$).
These results indicate that both the content and structure of duplicated genetic material contribute to facilitating adaptive evolution.

Interestingly, unlike our long-genome control, we did not observe an adaptive benefit from the nop-insert slip-duplication treatment.
One possible explanation for this difference is the long-genome control's inability to rapidly shrink genome size through slip deletions.
Indeed, in preliminary experiments we found that deleterious mutational load associated with larger genome size frequently drove genome shrinkage, necessitating enforcement of a lower bound on genome size.
This pattern aligns with established theory that contingent factors play an important role in ensuring preservation of new genome content \citep{Innan2010}.

% As above, earlier-reported experiments within our study system concur with present work, finding sequence content and order to both also benefit adaptive evolution within changing environments \citep{lalejini2017gene}.

% \subsection{How Does Slip Duplication Facilitate Adaptive Evolution?}

% Thus far, we have established that slip-duplications of genetic material can significantly benefit adaptive evolution, and that sequence content and order both contribute to this effect.
% We next sought to explain \textit{how} the structural characteristics of slip duplications promote adaptive evolution in our study system.

% In these investigations, we evaluated three hypotheses around evolutionary dynamics of slip duplications.
% First, we explored how the facilitation of adaptive evolution by gene duplication differed between simple and complex traits promoted by gene duplication.
% Second, we tested for signatures of evolutionary potentiation by gene duplications along lineage histories.
% Finally, we examined how slip duplication influences genetic architecture with respect to brittleness and vestigial genetic material.

\subsection{Duplicated sites are potentiated for complex traits}

% https://github.com/chaynes2019/AvidaGeneDupe/blob/4c7fa27229094adcb5bdb0b1aec541d0014b0fed/binder/hard-task-gain.ipynb
%             H-statistic       p-value
% Components
% 1              1.011820  3.144672e-01
% 2             22.787798  1.809107e-06
% 3             33.846753  5.962854e-09
% 4             15.359894  8.885441e-05
% 5              9.097744  2.559249e-03
%    Components  Prev Slip Insertion Cumulative Count      mean       std
% 0           1                                 False  0.932088  0.455109
% 1           1                                  True  1.002025  1.164197
% 2           2                                 False  0.588566  0.618479
% 3           2                                  True  1.579974  1.191681
% 4           3                                 False  0.491481  0.700256
% 5           3                                  True  1.516115  0.911430
% 6           4                                 False  0.703793  0.980105
% 7           4                                  True  1.229158  0.654042
% 8           5                                 False  0.583674  0.927917
% 9           5                                  True  1.249985  0.531423
%    Components  Prev Slip Insertion Cumulative Count  size
% 0           1                                 False    60
% 1           1                                  True    48
% 2           2                                 False    60
% 3           2                                  True    56
% 4           3                                 False    60
% 5           3                                  True    59
% 6           4                                 False    52
% 7           4                                  True    52
% 8           5                                 False    20
% 9           5                                  True    20

\begin{figure}
\includegraphics[
width=\linewidth
]{binder/binder/teeplots/density-norm=width+hue=prev-slip-insertion-cumulative-count+kind=violin+viz=catplot+x=components+y=is-task-coding-site+ext=.pdf}
\caption{%
  \textbf{TODO.}
  \footnotesize
  TODO
} \label{fig:potentiation}
\end{figure}



Thus far, we have established that slip duplication can promote evolution of novel traits, with this effect biasing toward complex traits.
% We next sought to characterize aspects of genetic architecture through which slip duplication drives adaptation.
We next sought to understand the extent to which duplicated genetic material itself contributes to potentiating novel neofunctional traits.

To address the question, we assessed the density of coding sites for novel tasks in genome regions that had previously been slip duplicated.
Figure \ref{fig:potentiation} compares involvement in coding for new tasks between previously slip-duplicated and non-slip-duplicated sites.
For the simplest tasks, requiring only one NAND component, we found no significant difference in the likelihood of duplicated sites participating in coding regions for new tasks.
% https://github.com/chaynes2019/AvidaGeneDupe/blob/2d31c0bd0cc9bdbf4bb49224859d6bd165cd36c8/binder/hard-task-gain2.ipynb
However, we found significant associations for traits with two or more NAND components (one-sample Wilcoxon signed-rank tests; $n=56,59,52,20$ observations).
Effect sizes of potentiation on likelihood to code for novel traits were $1.6\times$, $1.6\times$, $1.2\times$, and $1.2\times$, respectively, for 2, 3, 4, and 5 task components.
Smaller effect sizes at 4- and 5-component tasks may be due to a larger portion of the genome becoming comprised of slip-duplicated sites (Supplementary Figure \ref{fig:potentiation-supp}), thus lowering the upper ceiling on deviation from expected.

One possible factor confounding our assay for potentiation is evolutionary constraint at genome sites involved in organsims' self-replication loop.
These sites are critical to viability, with lethal outcomes when knocked out.
We found that these critical sites were less likely to be involved in slip duplication and also less likely to be involved in coding for \textit{de novo} traits.
% Hence, a spurious correlation could be introduced.
However, after excluding such fitness-critical sites from analysis, we still found generally similar potentiation signatures from slip duplication (Supplemental Figure \ref{fig:potentiation-supp}).

In addition to neofunctionalization, gene duplications are also hypothesized to directly facilitate adaptation by directly producing beneficial mutational changes, for example, through dose effects \citep{kondrashov2012gene}.
% https://github.com/chaynes2019/AvidaGeneDupe/blob/61ea29d989ebc8ad83dcfee94f3fa556b81e3f78/binder/gain-mechanism.ipynb
In line with this possibility, we observed that a substantial fraction of gain-of-function steps on lineages directly coincided with slip duplications --- 41 of 174, or 23.6\%.
However, in these cases, we still found evidence that sites coding for a new trait directly were more likely than chance to have been involved in earlier slip duplications (Supplemental Figure \ref{fig:potentiation-supp}).
Thus, the adaptive characteristics of slip duplication observed in our system seem likely to result from a combination of potentiation and direct facilitation.

\subsection{Genome length drives genetic complexity}

\begin{figure}
    \centering
    \begin{subfigure}{\linewidth}
    \includegraphics[width=\linewidth]{binder/binder/teeplots/hue=treatment+post=plt-xlim-0-600+viz=lineplot+x=generation-born+y=num-coding-sites+ext=.pdf}
    \caption{\footnotesize active coding sites}
    \label{fig:num-coding-sites:active}
    \end{subfigure}

    \begin{subfigure}{\linewidth}
\includegraphics[width=\linewidth,clip, trim=0 0 0 0.8cm]{binder/binder/teeplots/hue=treatment+post=plt-xlim-0-600+viz=lineplot+x=generation-born+y=num-free-sites+ext=.pdf}
    \caption{\footnotesize vestigial coding sites}
    \label{fig:num-coding-sites:vestigial}
    \end{subfigure}
    \caption{
        \textbf{Gene duplication boosts accumulation of vestigial coding sites.}
        \footnotesize
        Generation-by-generation counts of coding sites over evolutionary history.
        Here, ``active'' coding sites refer to genome instructions determined through knockout to contribute to fitness with respect to self-copy viability or a rewarded phenotypic trait.
        As shown in panel \ref{fig:num-coding-sites:active}, gene duplication yields active coding site counts comparable to long-genome control.
        Vestigial coding site count, by contrast, reports the number of sites determined to have contributed to fitness in an ancestor, but are no longer active coding sites.
        As shown in panel \ref{fig:num-coding-sites:active}, vestigial coding site count under slip-duplication treatment outpace control treatments.
        Error bands give 95\% CI, bootstrapped over 30 replicates per treatment.
    }
    \label{fig:num-coding-sites}
\end{figure}


In a final set of analyses, we broadened our scope to assess consequences of gene duplication on whole-genome architecture with respect to genome robustness, which we defined in terms of sensitivity of fitness to mutation \citep{lenski1999genome}.
We quantified robustness by counting the number of ``critical'' genome sites, where a single-site knockout disrupted replicator viability or one or more adaptive phenotypic traits (i.e., Logic-9 tasks).%
\footnote{%
Although sufficiently representative for our purposes, limitations exist in detecting Avida genome functionality through single-site knockouts; such an approach can underestimate aspects of genome sequence complexity involving small effects or redundancy \citep{lenski1999genome,moreno2024cryptic}.
}

One conventional perspective on gene duplication is \textit{vis-a-vis} neutral dynamics, wherein copied genetic material reduces brittleness by introducing redundancy \citep{wagner1996genetic}.
To assess the relevance of this model within our study system, we performed slip-duplicate mutational assays to quantify the baseline effect of slip insertion mutations on robustness, in the absence of selection.

We applied our assay to final-dominant genome lineages evolved with slip duplication.
For this purpose, we generated one slip duplication per genome, and filtered away those that affected self-replication viability or altered the Logic-9 phenotype.
We then measured change in critical site counts between corresponding wildtype and mutated variants.

In line with evidence that duplication-associated redundancy can boost robustness \citep{Lynch2000}, Fitness-neutral slip insertions did indeed tend to reduce the number of genome sites detectable as a single point of failure for performed tasks.
% https://github.com/chaynes2019/AvidaGeneDupe/blob/9fee1f13a8d31f25d1dd799c6a26f9b7fa617738/binder/indel-effect-nulldist.ipynb
On average, we found that neutral slip insertions decreased coding site count by 6.8 sites (bootstrapped 95\% CI 6.4 to 7.3).
This effect was strongest in genomes with high complexity; for instance, neutral insertion mutations decrease coding site count by 9.2 and 8.3 sites on average in genomes that encode 4- and 5-component complexity tasks, respectively (bootstrapped 95\% CIs 8.5 to 9.9 and 7.2 to 9.3).
Supplementary Figure \ref{fig:nulldist} presents these results.

% https://github.com/chaynes2019/AvidaGeneDupe/blob/binder/binder/gain-mechanism.ipynb
To assess evolutionary consequences of redundancies introduced by slip-duplication, we next pivoted to assess coding site accumulation within genomes over the course of evolution.
Counter to naive expectation, we found that the slip duplication treatment accrued fitness-critical sites at a generation-on-generation rate comparable to the long-genome baseline treatment (Figure \ref{fig:num-coding-sites:coding}; Mann-Whitney test, $U=361$, $p=0.19$).
Despite this similarity, however, when measurements were taken inclusive of vestigial coding sites (those which had \textit{previously} been fitness-critical earlier within a lineage), we found a significantly increased coding site count associated with the slip-duplicate treatment (Mann-Whitney test, $U=630$, $p<0.01$).
Figure \ref{fig:num-coding-sites:coded} compares vestigial-inclusive coding site counts trajectories along final-dominant lineages.
It therefore appears that slip duplication processes within our study system act to increase the net supply of coding material in the genome available to neutral processes, but neither suppress nor accelerate accumulation of contingent genetic brittleness.

% Indeed, in nature, large-scale genomic analyses have discovered cases where a strikingly high fraction of the genes in an organism show evidence of having arisen from gene duplications \citep{teichmann_structural_1998,Teichmann:2004cz}.
% Comparative studies have associated duplication events early in natural history with increases in genetic robustness and evolutionary innovation \citep{wagner_gene_2008}.

% One possible explanation is that selective pressures are driving the two treatments to reach a similar number of critical sites, but slip duplication can duplicate and copy genome content in a manner that increases the copy count of previously coding sites relative to the number of critical sites.
